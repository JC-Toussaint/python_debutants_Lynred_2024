\documentclass[a4paper, 12pt]{article}
\usepackage{a4wide}
\usepackage {amsmath}
\usepackage{amssymb}
\usepackage {graphicx}
\usepackage[utf8]{inputenc} 
\usepackage[francais]{babel}
\usepackage{fancyhdr}
\usepackage{setspace}
\usepackage{fancyhdr}
\usepackage{lastpage}
\usepackage{extramarks}
\usepackage{chngpage}
\usepackage{soul}
\usepackage{algorithmicx} 
\usepackage{algpseudocode} 
\usepackage{multicol}
\usepackage[usenames,dvipsnames]{color}
\usepackage{graphicx,float,wrapfig}
\usepackage{ifthen}
\usepackage{listings}
\usepackage{courier}
\usepackage{esint}
\usepackage{bbm}
\usepackage{graphics}
\usepackage{graphicx}
\usepackage{subfig}
\usepackage{epsfig}
\usepackage{pgf,tikz}
\usetikzlibrary{arrows}
\usepackage{braket}
\usepackage{MnSymbol,wasysym}
\usepackage{marvosym}
\usepackage{dsfont}
\usepackage{listings}
\usepackage{color}

\definecolor{mygreen}{rgb}{0,0.6,0}
\definecolor{mygray}{rgb}{0.5,0.5,0.5}
\definecolor{mymauve}{rgb}{0.58,0,0.82}

\lstset{ %
  backgroundcolor=\color{white},   % choose the background color; you must add \usepackage{color} or \usepackage{xcolor}; should come as last argument
  basicstyle=\footnotesize,        % the size of the fonts that are used for the code
  breakatwhitespace=false,         % sets if automatic breaks should only happen at whitespace
  breaklines=true,                 % sets automatic line breaking
  captionpos=b,                    % sets the caption-position to bottom
  commentstyle=\color{mygreen},    % comment style
  deletekeywords={...},            % if you want to delete keywords from the given language
  escapeinside={\%*}{*)},          % if you want to add LaTeX within your code
  extendedchars=true,              % lets you use non-ASCII characters; for 8-bits encodings only, does not work with UTF-8
  frame=single,	                   % adds a frame around the code
  keepspaces=true,                 % keeps spaces in text, useful for keeping indentation of code (possibly needs columns=flexible)
  keywordstyle=\color{blue},       % keyword style
  language=Python,                 % the language of the code
  morekeywords={*,...},           % if you want to add more keywords to the set
  numbers=left,                    % where to put the line-numbers; possible values are (none, left, right)
  numbersep=5pt,                   % how far the line-numbers are from the code
  numberstyle=\tiny\color{mygray}, % the style that is used for the line-numbers
  rulecolor=\color{black},         % if not set, the frame-color may be changed on line-breaks within not-black text (e.g. comments (green here))
  showspaces=false,                % show spaces everywhere adding particular underscores; it overrides 'showstringspaces'
  showstringspaces=false,          % underline spaces within strings only
  showtabs=false,                  % show tabs within strings adding particular underscores
  stepnumber=1,                    % the step between two line-numbers. If it's 1, each line will be numbered
  stringstyle=\color{mymauve},     % string literal style
  tabsize=2,	                   % sets default tabsize to 2 spaces
  title=\lstname                   % show the filename of files included with \lstinputlisting; also try caption instead of title
}

\newtheorem{de}{Définition}[subsection] % les definitions et les theoremes sont
\newtheorem{theo}{Theoreme}[section]    % numerotes par section
\newtheorem{prop}[theo]{Proposition}    % Les propositions ont le meme compteur
                                        % que les theoremes

\DeclareMathOperator{\tr}{tr}

\lhead{} 
\chead{} 
\rhead{\bfseries Python} 
\lfoot{JC Toussaint - Phelma }
%\cfoot{} 
%\rfoot{\thepage}

% Includes a MATLAB script.
% The first parameter is the label, which also is the name of the script
%   without the .m.
% The second parameter is the optional caption.
\newcommand{\matlabscript}[2]
  {\begin{itemize}\item[]\lstinputlisting[caption=#2,label=#1]{#1.m}\end{itemize}}

\lhead{} 
\chead{} 
\rhead{\bfseries Python} 
\lfoot{JC Toussaint - Phelma }

\pagestyle{fancy}

\begin{document}

\bibliographystyle{alpha}

\title{02 - Le projet tri rapide d'une liste}

\author{Lionel Bastard \& Jean-Christophe Toussaint\\
%  Phelma\\
 \texttt{prenom.nom@phelma.grenoble-inp.fr}}
\date{\today}
 
\maketitle


%L'exercice, proposé ci-après, est à réaliser en totale autonomie et constitue le socle
%de base en Python que vous devez assimiler.

\section{Introduction}

On génère une liste aléatoire $A$  de $N=1000$ entiers que l'on désire
trier par ordre croissant. 

\begin{enumerate} 
\item Utiliser la  fonction  numpy {\tt np.random.randint} pour générer les nombres.
Que remarquez-vous quand vous exécutez deux fois cette fonction?

\item Faire la même chose en la précédant de l'instruction 
{\tt np.random.seed(12345)} qui initialise le générateur de nombres aléatoires avec une graine.
Même question que précédemment.
\end{enumerate} 

\section{Tri par bulle}
\begin{enumerate} 
\item Dévélopper une fonction {\tt Tri\_bulle(A)} permettant de trier le tableau de manière croissante.
La méthode la plus simple dite de "tri par bulle" consiste à permuter si nécessaire, deux éléments successifs
du tableau et à recommencer dès le début du tableau tant qu'il n'est pas trié.

\begin{algorithmic}[1]
   \begin{multicols}{2}
\Function{Tri\_bulle}{$T$}
\State $n \gets \Call{Len}{$T$}$
\State $change \gets True$
    \While{$change == True$}
   \State $change \gets False$ 
   \For{$i \gets 0 {\; \bf to\; } n-1$}   
   \State $j \gets i+1$
    \If {$T[i]>T[j]$}
           \State $change \gets True$ 
           \State $T \gets \Call{Echanger}{$T, i, j$} $ \Comment{$T[i] \leftrightarrow T[j]$}
    \EndIf
    \EndFor
\EndWhile
    \State \Return $T$
\EndFunction
\end{multicols}
\end{algorithmic}
   Noter que le symbole $\rhd$ désigne le début d'un commentaire.
    
\item On désire mesurer le temps CPU nécessaire pour faire le tri. Utiliser la portion de code suivant :
\begin{lstlisting}[language=python]
from time import process_time

# Start the stopwatch / counter 
t1_start = process_time() 
   
A=tri_bulle(A)
  
# Stop the stopwatch / counter
t1_stop = process_time()
print("Elapsed time (s):", t1_stop-t1_start) 
\end{lstlisting}

Etudier l'évolution du temps de calcul en fonction de $N = 2^n$ pour $n \in \mathbb{N}$.
Tracer le temps de calcul en fonction de $N$ en représentation loglog.
Pour cette étude  de performance, on initialisera le générateur avec la même graine pour générer
toujours le même tableau pour une taille donnée. \\L'autre possibilité serait de faire une étude
statistique sur un grand nombre d'échantillons de tableaux de même taille.

 
\item  Hoare a inventé en 1962, une méthode de type "diviser pour régner"  permettant de trier un
tableau en $N \ln N$ opérations. 
Son algorithme s'écrit :

\begin{algorithmic}[1]
   \begin{multicols}{2}
\Function{Tri\_rapide}{$T,g,d$}
    \If {$g < d$}
        \State $ip, T \gets \Call{Partition}{$T, g, d$} $
         \State $T \gets \Call{Tri\_rapide}{$T, g, ip$} $
         \State $T \gets \Call{Tri\_rapide}{$T, ip+1, d$} $
    \EndIf
    \State \Return $T$
\EndFunction
\\
\Function{Partition}{$T,g,d$}
    \State $pivot \gets T[g]$
    \State $i \gets g-1$
    \State $j \gets d+1$
        \columnbreak
    \While{$true$}
    \Repeat
    \State $j \gets j-1$
    \Until{$T[j] <= pivot$}

    \Repeat
    \State $i \gets i+1$
    \Until{$T[i] >= pivot$}
    \If {$i < j$}
    \State $T \gets \Call{Echanger}{$T, i, j$} $ \Comment{$T[i] \leftrightarrow T[j]$}
    \Else
    \State \Return $j, T$
    \EndIf
        \EndWhile
\EndFunction
\end{multicols}
\end{algorithmic}

Noter que le symbole $\rhd$ désigne le début d'un commentaire.

\item La boucle conditionnelle {\tt repeat ... until (condition)} n'existe pas en Python, on vous demande de proposer une implémentation possible. 
On prend l'exemple d'une boucle incrémentale gérée par une variable $i$ variant
de 0 à 9 avant l'incrémentation.

\newpage 

\begin{algorithmic}[1]
    \State $i \gets 0$
    \Repeat \Comment{afficher \; $i$}
    \State $i \gets i+1$  
    \Until{$i==10$}
\end{algorithmic}

Cet algorithme peut être traduit en python par:
\begin{lstlisting}[language=python]
i=0
while True:
	print(i)
	i=i+1
	if i==10:
		break
\end{lstlisting}
Tester cette implémentation dans un programme à part.

\item Implémenter l'algorithme suivant sous la forme d'une fonction Python. 
On rappelle que l'échange des valeurs de deux variables $a \leftrightarrow b$ en python s'écrit :
$a,\; b=b,\; a$. Cette instruction fonctionne aussi pour 2 éléments (i, j) d'un tableau numpy :
$T[i], T[j]= T[j], T[i]$. Une version plus optimisée s'écrit : $T[[i, j]]=T[[j, i]]$

\item Son appel 
tri\_rapide(A, 0, len(A)-1) trie le tableau $A$ in-place.


%\item Faites un programme appelant la fonction tri\_rapide permettant de trier des tableaux
%de taille $2^n$ où $n \in [1, 20]$.  En utilisant la fonction {\tt \%timeit} , stocker le temps CPU
%dans un tableau et finalement tracer l'évolution du temps de calcul en fonction de la taille.
%
%\item Vérifier qu'il varie en $N \ln N$ dans la limite des grandes valeurs de $N$.

%\begin{figure}[h]
%   \caption{\label{Algo} Algorithme Tri Rapide}
%   \includegraphics[width=8cm]{Algo_Tri_Rapide}
%\end{figure}

\item Etudier et tracer l'évolution du temps de calcul en fonction de $N$.
Comparez avec la méthode précédente.

\end{enumerate} 

\end{document}

